\documentclass{article}
\usepackage[french]{babel}
\usepackage[utf8]{inputenc}
\usepackage{alltt,theorem}
\usepackage[tikz]{mdframed}

%%%%%%%%%% Start TeXmacs macros
\catcode`\<=\active \def<{
\fontencoding{T1}\selectfont\symbol{60}\fontencoding{\encodingdefault}}
\catcode`\>=\active \def>{
\fontencoding{T1}\selectfont\symbol{62}\fontencoding{\encodingdefault}}
\newcommand{\tmem}[1]{{\em #1\/}}
\newcommand{\tmfolded}[2]{\trivlist{\item[$\bullet$]\mbox{}#1}}
\newcommand{\tmstrong}[1]{\textbf{#1}}
\newcommand{\tmsubtitle}[1]{\thanks{\textit{Subtitle:} #1}}
\newcommand{\tmtextbf}[1]{{\bfseries{#1}}}
\newcommand{\tmtexttt}[1]{{\ttfamily{#1}}}
\newenvironment{itemizeminus}{\begin{itemize} \renewcommand{\labelitemi}{$-$}\renewcommand{\labelitemii}{$-$}\renewcommand{\labelitemiii}{$-$}\renewcommand{\labelitemiv}{$-$}}{\end{itemize}}
\newenvironment{tmcode}[1][]{\begin{alltt} }{\end{alltt}}
{\theorembodyfont{\rmfamily\small}\newtheorem{exercise}{Exercise}}
\mdfsetup{linecolor=black,linewidth=0.5pt,skipabove=0.5em,skipbelow=0.5em,hidealllines=true,
innerleftmargin=0pt,innerrightmargin=0pt,innertopmargin=0pt,innerbottommargin=0pt}
\newcounter{nnremark}
\def\thennremark{\unskip}
{\theorembodyfont{\rmfamily}\newtheorem{remark*}[nnremark]{Remark}}
{\theorembodyfont{\rmfamily\small}\newtheorem{solution}{Solution}}
\newmdenv[hidealllines=false,innertopmargin=1ex,innerbottommargin=1ex,innerleftmargin=1ex,innerrightmargin=1ex]{tmornamented}
%%%%%%%%%% End TeXmacs macros

\begin{document}

{\setthispageheader{}}\title{
  Variables, affectations
  \tmsubtitle{Chap. 3}
}

\maketitle

\hrulefill

{\tableofcontents}

\hrulefill

\section{Variables}

\begin{tmornamented}[roundcorner=1.7ex]
  Une variable est une zone de la mémoire repérée par un
  {\tmstrong{identificateur}}. Cet identificateur permet de {\tmem{modifier ou
  de faire appel au contenu de cette zone de la mémoire}} lors du
  déroulement du programme.
\end{tmornamented}

\begin{remark*}
  Le langage Python est {\tmem{sensible à la casse}}.
  {\python{ma\_variable}}, {\python{maVariable}} et {\python{mavariable}} sont
  donc {\tmem{trois identificateurs différents}} (préférer la première
  écriture).
\end{remark*}

\paragraph{Convention.}

\begin{itemizeminus}
  \item Le premier caractère de l'identificateur doit être une lettre ou un
  blanc souligné ({\hspace{0.17em}}\_{\hspace{0.17em}}).
  
  \item Les caractères suivant peut être alphanumériques ou des blanc
  soulignés ({\hspace{0.17em}}\_{\hspace{0.17em}}).
  
  \item Les majuscules et les minuscules ne sont pas confondues.
  
  \item Le tableau \ref{table:02}, contient les mots-clés du langage Python.
  {\tmem{Il est impossible de choisir un identificateur correspondant à l'un
  de ces mots-clés}}.
\end{itemizeminus}
\begin{table}[h]
  \begin{center}
    \begin{tabular}{ccccccccc}
      \hline
      and & as & assert & break & class & continue & def & del & elif\\
      \hline
      else & except & exec & finally & for & from & global & if & import\\
      \hline
      in & is & lambda & nonlocal & not & or & pass & print & raise\\
      \hline
      return & try & while & with & yield & None & True & False & \\
      \hline
    \end{tabular} 
  \end{center}
  \caption{Mots-clés réservés en Python\label{table:02}}
\end{table}

\section{Déclaration et affectation des variables}

\subsection{Vocabulaire}

\begin{tmornamented}[roundcorner=1.7ex]
  La \tmtextbf{déclaration} d'une variable consiste à réserver un
  emplacement dans la mémoire pour une utilisation ultérieure.
\end{tmornamented}

\paragraph{Note.} La déclaration d'une variable doit donc précéder son
utilisation.

\begin{tmornamented}[roundcorner=1.7ex]
  L'\tmtextbf{affectation} est une opération au cours de laquelle on copie
  dans une variable une {\tmem{valeur littérale}}, le {\tmem{contenu d'une
  autre variable}} ou le {\tmem{résultat d'un calcul}}.
\end{tmornamented}

\subsection{Affectation d'une valeur à une variable}

\subsubsection{Affectation simple}

\begin{tmornamented}[roundcorner=1.7ex]
  En Python, il est impossible de déclarer une variable sans l'utiliser.
  {\tmem{La variable est créée lorsqu'on lui affecte une donnée}}. On
  utilise à cet effet l'opérateur {\python{=}}.
  
  Python est un {\tmstrong{langage à typage dynamique}} : {\tmem{il n'est pas
  nécessaire de déclarer le type de la valeur référencée par la
  variable}}.
\end{tmornamented}

\begin{widetabular}
  {\tmstrong{En Python}}
  
  {\pythoncode{a = 3}} & {\tmstrong{En Java, C, C++, etc.}}
  
  \begin{tmcode}[cpp]
  int a;
a = 3;
  \end{tmcode}
\end{widetabular}

\paragraph{Remarque.} On peut noter dès à présent qu'une {\tmem{affectation
en Python ne consiste pas au stockage d'une valeur dans une variable}},
contrairement à ce qui se passe dans d'autres langages de programmation. En
Python, les objets (entiers, flottants, chaînes de caractères, ...) sont
\tmtextbf{référencés} : {\tmem{lors d'une affectation, c'est une
référence à un objet (et non une valeur) qui est définie}}, que l'objet
vienne d'être créé ou qu'il s'agisse d'un objet préexistant.

\begin{center}
  {\tmstrong{Une variable est, en Python, une étiquette qui permet d'accéder
  à une valeur.}}
\end{center}

{\pythoncode{>>> un\_entier = 12

>>> une\_chaine = 'cartable'

>>> un\_flottant = -2 + 3.12 + 17}}

\subsubsection{Affectations multiples}

{\pythoncode{>>> x = y = z = 1

>>> x

1

>>> y

1

>>> z

1}}

\paragraph{Remarque.} Pour bien comprendre comment fonctionne l'affectation
multiple, ci-dessus, {\tmstrong{lire l'instruction de la droite vers la
gauche}}. Ainsi, on affecte à la variable {\python{z}} l'entier 1, à la
variable {\python{y}} la valeur référencée par la variable {\python{z}}, à
la variable {\python{x}} la valeur référencée par la variable {\python{y}}.
L'instruction précédente est donc équivalente à l'enchaînement :

{\pythoncode{>>> z = 1

>>> y = z

>>> x = y}}

\subsubsection{Affectation de « tuples»}

{\pythoncode{>>> a, b, c = 1, 2, 3

>>> a

1

>>> b

2

>>> c

3}}

\paragraph{Remarque.} Le fonctionnement de cette affectation sera détaillé
dans un prochain chapitre.

\subsection{Opérateurs d'affectation}

Le langage Python possède un certain nombre de raccourcis permettant de très
rapidement écrire la mise à jour de la valeur d'une variable à partir d'une
expression faisant elle même référence au contenu de la variable.
{\tmem{Ici aussi, il est nécessaire de lire l'instruction de la droite vers
la gauche.}}

\begin{table}[h]
  \begin{center}
    \begin{tabular}{c|l|l}
      \hline
      \tmtextbf{Opérateur} & \tmtextbf{Expression} & \tmtextbf{Description}\\
      \hline
      {\python{=}} & {\python{variable = expression}} & \\
      {\python{+=}} & {\python{variable += expression}} & {\python{variable =
      variable + expression}}\\
      {\python{-=}} & {\python{variable -= expression}} & {\python{variable =
      variable - expression}}\\
      {\python{*=}} & {\python{variable *= expression}} & {\python{variable =
      variable * expression}}\\
      {\python{/=}} & {\python{variable /= expression}} & {\python{variable =
      variable / expression}}\\
      {\python{//=}} & {\python{variable //= expression}} & {\python{variable
      = variable // expression}}\\
      {\python{\%=}} & {\python{variable \%= expression}} & {\python{variable
      = variable \% expression}}\\
      \hline
    \end{tabular} 
  \end{center}
  \caption{Opérateurs d'affectation}
\end{table}

\section{Formatage des chaînes de caractères}

Comme nous l'avons vu dans le chapitre 2, une chaîne de caractère est un
type pré-défini incontournable du langage Python.

{\pythoncode{>>> type("abcd")

<class 'str'>

>>> type('abcd')

<class 'str'>

>>> type("Python, c'est super !")

<class 'str'>

>>> type('Python, c'est super !')

\ \ \ File "<pyshell>", line 1

\ \ \ \ \ \ type('Python, c'est super !')

\ \ \ \ \ \ \ \ \ \ \ \ \ \ \ \ \ \ \ \ \ \ \ \ \^{}

\ \ \ \ SyntaxError: invalid syntax

>>> type("2")

<class 'str'>

>>> type("a")

<class 'str'>

>>> type('-2.3')

<class 'str'>}}

Il faut donc être capable de manipuler ces chaînes de caractères, et en
particulier d'y incorporer le résultat de l'évalution
d'expressions\footnote{Nous en apprendrons bien plus sur les chaînes de
caractères dans un futur chapitre.}.

\begin{tmornamented}[roundcorner=1.7ex]
  La méthode {\python{chaine.format()}} permet d'insérer u{\tmem{n littéral
  ou le résultat de l'évaluation d'une expression}} dans une chaîne de
  caractères.
  
  Le type du littéral ou du résultat de l'évaluation est déterminé
  automatiquement ; la conversion vers le type {\python{str}} est alors
  effectuée.
\end{tmornamented}

{\pythoncode{>>> '\{\}, \{\}, \{\}'.format('a', 'b', 'c')

' a, b, c'

>>> '\{0\}, \{1\}, \{2\}'.format('a', 'b', 'c')

' a, b, c'

>>> "La somme de 1 + 2 est \{\}".format(1 + 2)

' La somme de 1 + 2 est 3'

>>> "La somme de \{1\} + \{2\} est \{0\}".format(2 + 3, 2, 3)

' La somme de 2 + 3 est 5'

>>> '\{0\}\{1\}\{0\}'.format('abra', 'cad')

' abracadabra'

>>> '\{:f\}; \{:f\}'.format(3.14, -3.14) \# :f est un marqueur

'3.140000; -3.140000'}}

Certaines mises en forme nécessitent de préciser le type de la donnée et la
façon de l'afficher. Le \tmtextbf{marqueur de formatage \tmtexttt{:}} permet
d'introduire des \tmtextbf{drapeaux} :
\begin{itemizeminus}
  \item \tmtexttt{s} indique que la donnée à insérer est une chaîne.
  {\tmem{C'est le drapeau implicite et il est inutile de le préciser}} ;
  
  \item le drapeau \tmtexttt{d} attend un nombre et le convertit en entier ;
  
  \item les drapeaux \tmtexttt{f}, \tmtexttt{g} et \tmtexttt{e} attendent des
  flottants. \tmtexttt{g} est le plus général, il correspond à \tmtexttt{f}
  si le nombre n'est pas trop grand, à la notation avec puissances de 10
  \tmtexttt{e}, si l'écriture du nombre prend beaucoup de place ;
  
  \item ...
\end{itemizeminus}
Pour finir, il est possible d'indiquer {\tmem{le nombre minimal de chiffres à
utiliser (ils n'ont pas besoin d'être visibles, seule la place nécessaire
pour le affichage compte) et le nombre de chiffres après la virgule à
afficher}}.

{\pythoncode{>>> "avant\{0:2f\}".format(3333.3333) \# au moins 2 chiffres
avant la virgule

' avant3333.333300' \ \ \ \ \ \ \ \ \ \ \ \ \ \ \ \ \ \# notation décimale

>>> "avant\{0:20f\}".format(3333.3333) \# au moins 20 chiffres avant la
virgule

' avant 3333.333300'

>>> "avant\{0:2.2f\}".format(3333.3333) \# au moins 2 chiffres avant et 2
chiffres après

' avant3333.33'

>>> "avant\{0:2.2f\}".format(3333.3383) \# au moins 2 chiffres avant et 2
chiffres après

' avant3333.34'

>>> "avant\{0:2.2e\}".format(3333.3333) \# au moins 2 chiffres avant et 2
chiffres après

'avant3.33e+03' \ \ \ \ \ \ \ \ \ \ \ \ \ \ \ \ \ \ \ \ \ \ \# notation
scientifique}}

\section{Exercices du chapitre}

\tmfolded{\begin{exercise}
  {\tmstrong{Variables}}
  
  Quelles sont les valeurs des nombres contenus dans les variables
  {\python{prix}}, {\python{tva}} et {\python{total}} après exécution de
  chacune des instructions suivantes ?
  
  {\pythoncode{>>> prix = 20
  
  >>> tva = 18.6
  
  >>> total = prix + prix * tva /100}}
\end{exercise}}{\begin{solution}
  \
  
  {\pythoncode{>>> prix
  
  20
  
  >>> tva
  
  18.6
  
  >>> total
  
  23.72}}
\end{solution}}

\tmfolded{\begin{exercise}
  {\tmstrong{Variables}}
  
  Quelles sont les valeurs des nombres contenus dans les variables
  {\python{prix}}, {\python{tva}} après exécution de chacune des
  instructions suivantes ?
  
  {\pythoncode{>>> prix = 20
  
  >>> tva = 18.6
  
  >>> prix = prix + prix * tva /100}}
\end{exercise}}{\begin{solution}
  \
  
  {\pythoncode{>>> prix
  
  23.72
  
  >>> tva
  
  18.6}}
\end{solution}}

\tmfolded{\begin{exercise}
  \
  
  Dans chacun des cas, quelles sont les valeurs des variables {\python{a}} et
  {\python{b}} après l'exécution de chacune des instructions suivantes ?
  
  \begin{widetabular}
    {\pythoncode{>>> a = 5
    
    >>> b = 7
    
    >>> b = a
    
    >>> a = b}} & {\pythoncode{>>> a = 5
    
    >>> b = 7
    
    >>> a = b
    
    >>> b = a}}
  \end{widetabular}
\end{exercise}}{\begin{solution}
  
  \begin{itemizeminus}
    \item Premier cas : {\python{a = 5}} et {\python{b = 5}}.
    
    \item Second cas : {\python{a = 7}} et {\python{b = 7}}
  \end{itemizeminus}
\end{solution}}

\tmfolded{\begin{exercise}
  {\tmstrong{Échange de valeurs}}
  
  Laquelle des instructions suivantes permet d'échanger les valeurs des deux
  variables {\python{a}} et {\python{b}} ?
  
  \begin{widetabular}
    {\pythoncode{>>> a = b
    
    >>> b = a}} & {\pythoncode{>>> t = a
    
    >>> a = b
    
    >>> b = t}} & {\pythoncode{>>> t = a
    
    >>> b = a
    
    >>> t = b}}
  \end{widetabular}
\end{exercise}}{\begin{solution}
  
  \begin{itemizeminus}
    \item Premier cas : {\python{a}} et {\python{b}} ont la valeur de
    {\python{b}}.
    
    \item Deuxième cas : {\python{a}} a la valeur de {\python{b}},
    {\python{b}} celle de {\python{a}}. C'est la bonne méthode !
    
    \item Troisième cas : {\python{a}}, {\python{b}} et {\python{t}} ont la
    valeur de a.
  \end{itemizeminus}
\end{solution}}

\tmfolded{\begin{exercise}
  {\tmstrong{Échange de valeurs}}
  
  Soit trois variables {\python{a}}, {\python{b}} et {\python{c}}. Écrire
  les instructions permutant les valeurs, de sorte que la valeur de
  {\python{a}} passe dans {\python{b}}, celle de {\python{b}} dans
  {\python{c}} et celle de {\python{c}} dans {\python{a}}. N'utiliser qu'une
  (et une seule) variable entière supplémentaire, nommée {\python{tmp}}.
\end{exercise}}{\begin{solution}
  \
  
  {\pythoncode{tmp = a
  
  a = c
  
  c = b
  
  b = a}}
\end{solution}}

\tmfolded{\begin{exercise}
  {\tmstrong{Conversion francs euros}}
  
  Écrire une fonction qui convertit des francs en euros (1 euro correspond à
  6,55957 francs).
  
  La spécification de la fonction est :
  
  {\pythoncode{def conversion\_franc\_euro(francs: float) -> str:
  
  \ \ \ """ Conversion d'une somme en francs en euros.
  
  \ \ \ Le message retourné donne les deux valeurs.
  
  \ \ \
  
  \ \ \ >>> conversion\_franc\_euro(100)
  
  \ \ \ '100 francs correspondent à 15.244901723741037 euros.'
  
  \ \ \ """}}
\end{exercise}}{\begin{solution}
  \
  
  {\pythoncode{def conversion\_franc\_euro(francs: float) -> str:
  
  \ \ \ """ Conversion d'une somme en francs en euros.
  
  \ \ \ Le message retourné donne les deux valeurs.
  
  \ \ \
  
  \ \ \ >>> conversion\_franc\_euro(100)
  
  \ \ \ '100 francs correspondent à 15.244901723741037 euros.'
  
  \ \ \ """
  
  \ \ \ change = 6.55957
  
  \ \ \ euros = francs / change
  
  \ \ \ return "\{\} francs correspondent à \{\} euros.".format(francs,
  euros)}}
\end{solution}}

\tmfolded{\begin{exercise}
  {\tmstrong{Message d'accueil}}
  
  Écrire une fonction qui, à partir d'un nom, d'un prénom et d'une date de
  naissance retourne un message d'accueil incluant l'age de la personne.
  
  La spécification de la fonction est :
  
  {\pythoncode{def salutation(nom: str, prenom: str, naissance: int) -> str:
  
  \ \ \ """ À partir des nom, prénom et date de naissance, retourne un
  
  \ \ \ message d'accueil indiquant l'âge.
  
  \
  
  \ \ \ >>> salutation("Dupond", "Patrick", 1961)
  
  \ \ \ 'Bonjour Patrick Dupond, vous avez 58 ans.'
  
  \ \ \ """}}
\end{exercise}}{\begin{solution}
  \
  
  {\pythoncode{def salutation(nom: str, prenom: str, naissance: int) -> str:
  
  \ \ \ """ À partir des nom, prénom et date de naissance, retourne un
  
  \ \ \ message d'accueil indiquant l'âge.
  
  \ \ \
  
  \ \ \ >>> salutation("Dupond", "Patrick", 1961)
  
  \ \ \ 'Bonjour Patrick Dupond, vous avez 58 ans.'
  
  \ \ \ """
  
  \ \ \ annee\_en\_cours = 2019
  
  \ \ \ message = "Bonjour \{0\} \{1\}, vous avez \{2\} ans.".format(prenom,
  
  \ \ \ \ \ \ \ \ \ \ \ \ \ \ \ \ \ \ \ \ \ \ \ \ \ \ \ \ \ \ \ \ \ \ \ \ \ \
  \ \ \ \ \ \ \ \ \ \ \ \ \ \ \ \ \ \ \ \ nom,
  
  \ \ \ \ \ \ \ \ \ \ \ \ \ \ \ \ \ \ \ \ \ \ \ \ \ \ \ \ \ \ \ \ \ \ \ \ \ \
  \ \ \ \ \ \ \ \ \ \ \ \ \ \ \ \ \ \ \ \ annee\_en\_cours - naissance)
  
  \ \ \ return message}}
\end{solution}}

\tmfolded{\begin{exercise}
  {\tmstrong{Nombre aléatoire}}
  
  Écrire une fonction qui, à partir de deux ages limites, retourne une
  chaîne de caractères indiquant l'age de la personne qui appelle cette
  fonction, choisi aléatoirement.
  
  La spécification de la fonction est :
  
  {\pythoncode{def age\_aleatoire(age\_min: int, age\_max: int) -> str:
  
  \ \ \ """ À partir de deux ages limites, retourne un message contenant
  
  \ \ \ un age déterminé aléatoirement.
  
  \ \ \
  
  \ \ \ >>> age\_aleatoire(20, 100)
  
  \ \ \ 'Votre age est : 85 ans !'
  
  \ \ \ """}}
\end{exercise}}{\begin{solution}
  \
  
  {\pythoncode{import random
  
  \
  
  def age\_aleatoire(age\_min: int, age\_max: int) -> str:
  
  \ \ \ """ À partir de deux ages limites, retourne un message contenant
  
  \ \ \ un age déterminé aléatoirement.
  
  \ \ \
  
  \ \ \ >>> age\_aleatoire(20, 100)
  
  \ \ \ 'Votre age est : 85 ans !'
  
  \ \ \ """
  
  \ \ \ age = random.randint(age\_min, age\_max)
  
  \ \ \ return "Votre age est : \{\} ans !".format(age)}}
\end{solution}}

\end{document}
